
\chapter{Introduction}
    \section{Purpose}
        \paragraph{}
The Bid Emission and Exchange Protocol is a protocol for a distributed
bid systems. It enables users using the client to bid for objects among a list. This
protocol is referred to as the BEEP/0.1a.

    \section{Requirements}
        \paragraph{}
The key words "MUST", "MUST NOT", "REQUIRED", "SHALL", "SHALL NOT",
"SHOULD", "SHOULD NOT", "RECOMMENDED", "MAY", and "OPTIONAL" in this
document are to be interpreted as described in RFC 2119 [34].
        \paragraph{}
An implementation is not compliant if it fails to satisfy one or more
of the MUST or REQUIRED level requirements for the protocols it
implements. An implementation that satisfies all the MUST
level requirements but not all the SHOULD level requirements for its
protocols is said to be "conditionally compliant"; one that satisfies
all the MUST or REQUIRED level and all the SHOULD level requirements
for its protocols is said to be "unconditionally compliant";
        \paragraph{}
The key words "MUST", "MUST NOT", "REQUIRED", "SHALL", "SHALL NOT",
"SHOULD", "SHOULD NOT", "RECOMMENDED", "MAY", and "OPTIONAL" in this
document are to be interpreted as described in RFC 2119 [34].

    \section{Terminology}
        \paragraph{}
This specification uses a number of terms to refer to the roles
played by participants in, and objects of, the BEEP communication.
        \paragraph{connection}
A transport layer virtual circuit established between two programs
in order to establish a communication.
        \paragraph{login IDs}
A combination of login and password.
        \paragraph{message}
The basic unit of BEEP communication, consisting of a structured
sequence of octets matching the syntax defined in section 4 and
transmitted via the previously established connection.
        \paragraph{request}
A BEEP request message, as defined in section 5.
        \paragraph{response}
A BEEP response message, as defined in section 6.
        \paragraph{resource}
A network data object or service that can be identified by a URI,
as defined in section 3.2. Resources may be available in multiple
representations (e.g. multiple languages, data formats, size, and
resolutions) or vary in other ways.
        \paragraph{entity}
The information transferred as the payload of a request or
response. An entity consists of meta-information in the form of
entity-header fields and content in the form of an entity-body, as
described in section 7.
        \paragraph{representation}
An entity included with a response that is subject to content
negotiation, as described in section 12. There may exist multiple
representations associated with a particular response status.
        \paragraph{content negotiation}
The mechanism for selecting the appropriate representation when
servicing a request, as described in section 12. The
representation of entities in any response can be negotiated
(including error responses).
        \paragraph{client}
A program that establishes connections for the purpose of sending
requests.
        \paragraph{server}
An application program that accepts connections in order to
service requests by sending back responses. Any given program may
be capable of being both a client and a server; our use of these
terms refers only to the role being performed by the program for a
particular connection, rather than to the program's capabilities
in general. Likewise, any server may act as an origin server,
proxy, gateway, or tunnel, switching behavior based on the nature
of each request.
        \paragraph{origin server}
The server on which a given resource resides or is to be created.
        \paragraph{semantically transparent}
A cache behaves in a "semantically transparent" manner, with
respect to a particular response, when its use affects neither the
requesting client nor the origin server, except to improve
performance. When a cache is semantically transparent, the client
receives exactly the same response (except for hop-by-hop headers)
that it would have received had its request been handled directly
by the origin server.
        \paragraph{upstream/downstream}
Upstream and downstream describe the flow of a message: all
messages flow from upstream to downstream.

    \section{Overall Operation}
        \paragraph{}
   BEEP is a request/response protocol. The client send a request to the
   server and the server answer with a response. All the messages
   (requests and responses) go through a connection between the server
   and the client. The connection MUST be established before any request
   other than those using the CONNECT method occurs.
   The request is composed of a method, an entity and the protocol
   version.
        \paragraph{}
   The response is composed of a status-line, response-headers and in
   some cases an entity.


    \clearpage
\chapter{Notational Conventions and Generic Grammar : Basic Rules}
    \paragraph{}
   The following rules are used throughout this specification to
   describe basic parsing constructs. The US-ASCII coded character set
   is defined by ANSI X3.4-1986 [21].
    \paragraph{}
       OCTET          = <any 8-bit sequence of data>
    \paragraph{}
       CHAR           = <any US-ASCII character (octets 0 - 127)>
    \paragraph{}
       DIGIT          = <any US-ASCII digit "0".."9">
    \paragraph{}
       INT            = <basic signed integer type. cf ISO/IEC 9899>
    \paragraph{}
       LONG INT       = <long signed integer type. cf ISO/IEC 9899>
    \paragraph{}
       FLOAT          = <floating-point type. cf IEEE 754>
    \paragraph{}
       SP             = <US-ASCII SP, space (32)>
    \paragraph{}
       CR             = <US-ASCII CR, carriage return (13)>
    \paragraph{}
       LF             = <US-ASCII LF, linefeed (10)>
    \paragraph{}
       LWS            = <US-ASCII LWS, linear white space (9)>
    \paragraph{}
       <">            = <US-ASCII double-quote mark (34)>


    \clearpage
\chapter{Protocol Parameters}
    \section{Date/Time Formats}
        \paragraph{Full Date}
      This application allows only one full date format :
         24-11-2014 14:41:12 GMT
      This is a format from the RFC 3339.
        \paragraph{Delta Seconds}
      Delta Seconds represents a difference between server's starting
      time and current time.
    \section{Character Sets}
        \paragraph{}
      BEEP uses the same definition of the term "character set" as that
   described for BEEP :
      The term "character set" is used in this document to refer to a
   method used with one or more tables to convert a sequence of octets
   into a sequence of characters. Note that unconditional conversion in
   the other direction is not required, in that not all characters may
   be available in a given character set and a character set may provide
   more than one sequence of octets to represent a particular character.
   This definition is intended to allow various kinds of character
   encoding, from simple single-table mappings such as US-ASCII to
   complex table switching methods such as those that use ISO-2022's
   techniques. However, the definition associated with a BEEP character
   set name MUST fully specify the mapping to be performed from octets
   to characters. In particular, use of external profiling information
   to determine the exact mapping is not permitted.
